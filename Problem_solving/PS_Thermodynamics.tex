\section{Problems}

\vspace{1cm}
\noindent {\bf Problem \thesection.\theprob}\stepcounter{prob} 

The turbomachines conveying air are classified usually as fans ($p_2/p_1<1.3$), blowers ($1.3<p_2/p_1<3$) and compressors ($3<p_2/p_1$). Assuming $p_1=1\,\mathrm{bar}$ inlet pressure, $t_1=20^o\,\mathrm{C}$ inlet temperature and isentropic process ($\kappa=1.4$), find the the relative density change $(\rho_2-\rho_1)/\rho_1$ at the fan-blower border and the $t_2$ outlet temperature at the blower-compressor border. (Solution: $(\rho_2-\rho_1)/\rho_1=20.6\%$, $t_2=128.1^o\,\mathrm{C}$)

\vspace{1cm}
\noindent {\bf Problem \thesection.\theprob}\stepcounter{prob} 

Assuming isentropic process of an ideal gas, find the inlet cross section area and the isotherm useful power of a compressor conveying $\dot m = 3\,\mathrm{kg/s}$ mass flow rate. The velocity in the inlet section is $c=180\,\mathrm{m/s}$. The surrounding air is at rest with $p_0=1\,\mathrm{bar}$ and $T_0=290\,\mathrm{K}$. $c_p=1000 \mathrm{J/kgK}$, $\kappa=1.4$. \add{The pressure at the outlet is equal to $p_2=4\,bar$.} (Solution: $A_1=0.016\,\mathrm{m^2}$)
%(Solution: $A_1=0.016\,\mathrm{m^2}$, $P_{isoth,u}=346.5\,\mathrm{kW}$)

\vspace{1cm}
\noindent {\bf Problem \thesection.\theprob}\stepcounter{prob} 

Gas is compressed from 1 bar absolute pressure to \change{3}{4} bar \textbf{relative} pressure. The gas constant is $288$J/kgK, the specific heat at constant pressure is $c_p = 1005\,J/kgK$. The exponent describing the polytropic compression is  $n = 1.54$. Find the isentropic exponent. Find the isentropic specific useful work, the specific input work and the isentropic efficiency. The density of atmospheric air is $1.16\, \mathrm{kg/m^3}$. $h_t\approx h$ is a reasonable approximation. 
(Solution: $\kappa=1.402$, $Y_{isentropic}=176.28$ kJ/kg, $Y_{input}=228.12$ kJ/kg, $\eta_{isentropic}=77.28$\%.)
%(Solution: $\kappa=1.402$, $Y_{isentropic}=146.729$ kJ/kg, $Y_{input}=188.289$ kJ/kg, $\eta_{isentropic}=77.9$\%.)

\vspace{1cm}
\begin{tcolorbox}
\noindent {\bf Problem \thesection.\theprob}\stepcounter{prob} 

Air is compressed from $1$ bar absolute pressure to $3$ bar relative pressure. The ideal gas constant is $287$ J/kgK. Calculate the temperature at the end of the compression, if the process is adiabatic, and the value of the heat capacity ratio is: $\kappa=1.4$. The air temperature at the inlet is $t_1 = 10^oC$. Calculate the input shaft work, if the losses are Y’ = $70$ kJ/kg. Find the isentropic useful work and the isentropic efficiency! The $h_{tot}\approx$ h approximation is reasonable because the change of the kinetic energy is negligible.
\vspace{0.2cm}

Solution:
\vspace{0.2cm}

$p_2=4 bar$ absolute pressure. At the end of the isentropic compression the temperature of the gas is: 
$\frac{T_{2s}}{T_1}=\left(\frac{p_2}{p_1}\right)^{\frac{\kappa-1}{\kappa}}=\left(\frac{4}{1}\right)^{\frac{1.4-1}{1.4}}$ = 1,486 from which we get: $T_{2s}=1.468T_1=420 K$. However, if we consider the losses, the temperature of the fluid will increase further.
The heat capacity at constant pressure:

\begin{equation*}
c_p=\frac{\kappa R}{\kappa-1}=\frac{\kappa R}{\kappa-1}=\frac{1.4\times287}{1.4-1}=1005 J/kgK.
\end{equation*}

\begin{equation*}
Y_{u,isentropic}=c_p(T_{2s}-T_1)=1005\times(420-283)=137.6 kJ/kg.
\end{equation*}
\vspace{0.1cm}

Next, we add the losses:

\begin{equation*}
Y_{in}=Y_{u,isentropic}+Y'=137.6 + 70 = 207.6 kJ/kg.
\end{equation*}
\vspace{0.1cm}

The isentropic efficiency is: $\eta_{isentropic}=\frac{137.6}{207.6}=0.66$. The temperature at the end of the compression:

\begin{equation*}
T_2=T_1+\frac{Y_{in}}{c_p}=283+\frac{207.6}{1.005}=489 K = 216^oC.
\end{equation*}

\end{tcolorbox}

\vspace{1cm}
\noindent {\bf Problem \thesection.\theprob}\stepcounter{prob} 

Ideal gas (gas constant $R=288$ J/kgK\add{, specific heat at constant pressure is  $c_p = 1005\,J/kgK$}) with $27^oC$ and 1 bar pressure is compressed to 3 bar with compressor. The exponent describing the real state of change is $n = 1.5$. Find the absolute temperature and density of the air at the outlet. Find the isentropic outlet temperature, the isentropic efficiency and the isentropic useful specific work. Find the power needed to cover the losses, if the mass flow is 3 kg/s. (Solution: $T_{real}=432.7$K, $\rho=2.407\,\mathrm{kg/m^3}$, $T_{isentropic}=410.6$K, $\eta_{isentropic}=83.3$\%,  $Y_{isentropic}=111.48$ kJ/kg, $P_{loss}=66.8$kW)

\vspace{1cm}
\noindent {\bf Problem \thesection.\theprob}\stepcounter{prob} 

Gas is compressed from 1 bar to 5 bar. The ambient air temperature at the inlet $t_1=22^{\circ}C$ while at the outlet $t_2=231^{\circ}C$. Gas constant $R=288$ J/kgK. Find the exponent describing the politropic compression and the density of air at the inlet and the outlet. (Solution: $n=1.5$, $\rho_1=1.177 \mathrm{kg/m^3}$, $\rho_2=3.44 \mathrm{kg/m^3}$.)

\vspace{1cm}
\noindent {\bf Problem \thesection.\theprob}\stepcounter{prob} 

Along a natural gas pipeline compressor stations are installed $L=75\,\mathrm{km}$ distance far from each other. On the pressure side of the compressor the pressure is $p_p=80\,\mathrm{bar}$, the density is $\rho_p=85\,\mathrm{kg/m^3}$, while the velocity of the gas is $v_p=6.4\,\mathrm{m/s}$. The diameter of the pipe is $D=600\,\mathrm{mm}$ the friction loss coefficient is $\lambda=0.018$. \textbf{Assuming that the process along the pipeline is isotherm}, the pressure loss is calculated as $\frac{p_{beg}^2-p_{end}^2}{2}=p_{beg}\lambda \frac{L}{D}\frac{\rho_{beg}}{2}v_{beg}^2$.

\begin{itemize}
\item Find the pressure, the density, and the velocity at the end of pipeline.
\item Find the mass flow through the pipeline.
\item Find the needed compressor power assuming that the compression is a politropic process and $n=1.45$.
\item Find the ratio of the compressor power and the power that could be released by the complete combustion of the transported natural gas. The heating value of the natural gas is $H=43\,\mathrm{MJ/kg}$. (Hint: $P_{comb}=\dot{m}H$)
\end{itemize}
%(Solution: $p_s=11.54\,\mathrm{bar}, \rho_s=12.26\,\mathrm{kg/m^3}$, $v_s=44.4\,\mathrm{m/s}$, $\dot{m}=153.8\,\mathrm{kg/s}$, $P_{comp}=38.42\,\mathrm{MW}$, $P_{comp}/P_{comb}=0.58\%$)

% \vspace{1cm}
% \noindent {\bf Problem \thesection.\theprob}\stepcounter{prob} 

% Air from $1\,\mathrm{bar}$ absolute pressure is compressed to $3\,\mathrm{bar}$ relative pressure. The gas constant is $288\,\mathrm{J/kgK}$, and the ration of the specific heat capacities of $\kappa=1.4$. Assuming an adiabatic and reversible process, find the temperature at the end of the compression if the temperature of the inflow air is $10^{\circ}C$? What is the input specific work, if the losses are $Y'=70\,\mathrm{kJ/kg}$? Find the actual temperature after the compression? (Solution: $T_{2,s}=420\,\mathrm{K}$, $Y_{in}=208.1\,\mathrm{kJ}$, $T_2=489\,\mathrm{K}$).


%\newpage

\vspace{1cm}
\begin{tcolorbox}
\noindent {\bf Problem \thesection.\theprob}\stepcounter{prob} 

A compressor carries air from a large open space to a tank (Figure \ref{gen_theprob}). The properties of the ambient fluid are the following: $T=27^{\circ}C,~ R=286\,\mathrm{J/kgK},~ \kappa=1.4$. The pressure after the compressor is $4\,\mathrm{bar}$, and the volumetric flow rate just before the compressor is $Q=2.5\,\mathrm{m/s}$. The politropic gas constant, which describes the compression  is $n=1.54$. The diameter of the pipe at the suction side and the pressure side is $125\,\mathrm{mm}$. Check if the Mach number at the suction side is lower than $0.7$!
(Solution: $Ma_1 = 0.609$, $Y_{in} = 200\,\mathrm{kJ/kg}$, $P_{in}=487\,\mathrm{kW}$, $Y_u = 158.2\,\mathrm{kJ/kg}$, $p_3=2.42\,\mathrm{bar}$, $T_{2,s}=445.9\,\mathrm{K}$, $Y_{isentropic} = 167.4\,\mathrm{kJ/kg}$, $Y_{isotherm}=131\,\mathrm{kJ/kg}$.
\vspace{0.2cm}

%\begin{figure}[ht]
%\begin{center}
%\includegraphics[scale=0.5]{figs/problem_1.5.9_compr_fig.png}
%\caption{\label{gen_term}System}
%\end{center}
%\end{figure}

Solution:
\vspace{0.2cm}

Check if the Mach number at the suction side is lower than $0.7$!

$0$–$1$ analysis of the isentropic process (the losses are neglected):
\begin{equation*}
p_0=1 bar, T_0=300K, R=286J/kgK, \kappa=1.4, Q_1=2.5m^3/s, d_1=d_2=125mm
\end{equation*}
Cross section of the inlet: $A_1={d_1}^2\pi/4=0.1252\times\pi/4=0.01227 m^2$.
Velocity at the inlet: $c_1=Q_1/A_1=2.5/0.01227=203.7m/s$.
Density of the ambient air: $\rho_0=p_0/RT_0=10^5/286/300=1.166 kg/m^3$ , $c_p=\frac{\kappa R}{\kappa-1}$.
The velocity of the air increases as an isentropic process as it enters the pipe at the inlet:
\begin{equation*}
h_0=h_{total}=const.=h_1+{c_1}^2/2, \text{therefore } T_1=T_0-{c_1}^2/(2c_p)=300-203.7^2/2/1001=279.3 K
\end{equation*}
The Mach-number at the local speed of sound is: $Ma_1=\frac{c_1}{\sqrt{\kappa RT_1}}$
\begin{equation*}
Ma_1=\frac{c_1}{\sqrt{\kappa RT_1}}=\frac{203.7}{\sqrt{1.4 \times 286 \times 279.3}} = 0.609<0.7
\end{equation*}
($0.7$  is a prescribed design parameter, which ensures that the Mach number is less than one.)

\end{tcolorbox}
\vspace{1.5cm}

\begin{figure}[ht]
\begin{center}
\includegraphics[scale=0.65]{figs/problem_1p5p9_compr_fig.png}
\caption{\label{gen_theprob}Compression system of Problem \thesection.\theprob.}
\end{center}
\end{figure}

\vspace{1cm}
\begin{tcolorbox}[breakable]
\noindent {\bf Problem \thesection.\theprob}\stepcounter{prob}

Find the input shaft work, the shaft power and the polytropic useful work of the previous problem! Calculate the pressure of the fluid in the tank after it cools down to the temperature of the
ambient fluid! Find the temperature assuming that the compression is an isentropic process! Neglecting the kinetic energy, find the useful work in case of an isentropic and isotherm process (the pressure
before and after the compressor can be assumed to be the same as for the politropic process).
\vspace{0.2cm}

Solution:
\vspace{0.2cm}

Calculate the input shaft work, the shaft power, and the polytropic useful work! The critical energy change of the air shall be considered.
\vspace{0.1cm}

$1$ – $2$: the compression can be approximated as a polytropic process.


Pressure of the air at the inlet: $p_1=p_0\left(\frac{T_1}{T_0}\right)^{\frac{\kappa}{\kappa-1}}=100 \cdot\left(\frac{279.3}{300}\right)^{\frac{1.4}{1.4-1}}=77.8 kPa$.
Density of the air at the inlet: $\rho_1=\frac{p_1}{RT_1}=279.3 \times \frac{77800}{286} \times 279.3=0.974 kg/m^3$.
The mass flow rate is: $\dot m=A_1\rho_1c_1=0.01227 \times 0.795 \times 203.7=2.436 kg/s$.


The temperature at the end of the compression is:
\begin{equation*}
	T_2=T_1\left(\frac{p_2}{p_1}\right)^{\frac{n-1}{n}}=279.3\left(\frac{4}{0.778}\right)^{\frac{1.54-1}{1.54}}.
\end{equation*}
The density of the air at the outlet of the compressor is: $\rho_2=\frac{p_2}{RT_2}=\frac{400000}{286 \times 495.5}=2.82 kg/m^3$.
The velocity of the air at the outlet of the compressor is: $c_2=\frac{\dot m}{A_2\rho_2}=\frac{2.436}{0.01227 \times 2.82}=70.4 m/s$.


The shaft work during the compression is:
\begin{equation*}
	Y_{in}=c_p(T_2-T_1)+\frac{{c_2}^2-{c_1}^2}{2}+g(z_2-z_1)=1001 \times (495.9-279.3)+\frac{{70.4}^2-{203.7}^2}{2}+0=199.9 kJ/kg.
\end{equation*}
The power of the compressor is:$P={\dot m}Y_{in}=2.436 \times 199.9=487 kW$.
The polytropic useful work is:
\begin{equation*}
	Y_{pol}=\frac{n}{n-1}\frac{p_1}{\rho_1}\left[\left(\frac{p_2}{p_1}\right)^{\frac{n-1}{n}}\right]=\frac{1.54}{1.54-1}\times\frac{77800}{0.974}\left[\left(\frac{4}{0.778}\right)^{\frac{1.54-1}{1.54}}\right]=176.7 kJ/kg.
\end{equation*}
As the air in the closed tank reaches its equilibrium state, is cools down to the ambient
temperature. Calculate the pressure in the tank after the air reaches the ambient temperature!
The cooling of the air in the tank (which is marked as number three in the figure) can be
approximated as an isochoric process (Gay-Lussac II. law, $p/T$ = const.)
\begin{equation*}
	p_3=p_2\frac{T_3}{T_2}=p_2\frac{T_0}{T_2}=400000 \times \frac{300}{495.7}=2.42 bar.
\end{equation*}
Calculate the useful work, if the compression process is assumed to be (i) isentropic or (ii)
isothermal. The pressure at the end of the compression should be the same as in the case of
the polytropic process!
Useful works: $p_1 -> p_2=4 bar$
Isentropic:
\begin{equation*}
	Y_{isentropic}=\frac{\kappa}{\kappa-1}\frac{p_1}{\rho_1}\left[\left(\frac{p_2}{p_1}\right)^{\frac{\kappa-1}{\kappa}}\right]=\frac{1.4}{1.4-1}\times\frac{77800}{0.974}\left[\left(\frac{4}{0.778}\right)^{\frac{1.4-1}{1.4}}\right]=166.8 kJ/kg.
\end{equation*}
Isothermal:
\begin{equation*}
	Y_{isothermal}=RT_1ln\left(\frac{p_2}{p_1}\right)=286 \times 279.3 \times ln\left(\frac{4}{0.778}\right)=130.8 kJ/kg
\end{equation*}
(which is the actual useful work, considering that the air cools down in the tank).
The shaft work is: $Y_in = 199,9 kJ/kg = 200 kJ/kg$. 
\end{tcolorbox}

\vspace{1cm}
\begin{tcolorbox}
\noindent {\bf Problem \thesection.\theprob}\stepcounter{prob}

Calculate the efficiency in the case of the different processes of the previous problem.
\vspace{0.2cm}

Solution:
\vspace{0.2cm}

From the value of the input/shaft work and
the useful work, the efficiency can be calculated in the case of the different processes.

\begin{equation*}
	\eta_{isothermal}=\frac{Y_{isothermal}}{Y_{in}}=\frac{130.8}{199.9}=0.654
\end{equation*}

\begin{equation*}
	\eta_{isentropic}=\frac{Y_{isentropic}}{Y_{in}}=\frac{166.8}{199.9}=0.834
\end{equation*}

\begin{equation*}
	\eta_{pol}=\frac{Y_{pol}}{Y_{in}}=\frac{176.4}{199.9}=0.882
\end{equation*}

In the figure below (Figure \ref{gen_fig}), the h-s diagram of dry air is displayed. The red curves are the isochore
($v=1/\rho=const.$) curves, while the black curves are the isobar ($p=const.$). The compression
process is displayed by the blue 0–1–2–3 lines.
\end{tcolorbox}
\vspace{1cm}

\begin{figure}[ht]
\begin{center}
\includegraphics[scale=0.75]{figs/problem_1p5p11_fig.png}
\caption{\label{gen_fig}h-s diagram of dry air}
\end{center}
\end{figure}	

\vspace{0.7cm}
\noindent {\bf Problem \thesection.\theprob}\stepcounter{prob} 

At the pressure side of the compressor $2.5\,\mathrm{bar}$ absolute pressure and $187^{\circ}C$ temperature was measured,  the temperature of the inflow air is $27^{\circ}C$, and the pressure at the suction side is $1020\,\mathrm{hPa}$ ($1\,\mathrm{hPa}=100\,\mathrm{Pa}$). Find the politropic exponent of the compression and the politropic efficiency of the process, if $\kappa=1.4$! (Solution: $n=1.911$, $\eta_{pol}=0.6$).
